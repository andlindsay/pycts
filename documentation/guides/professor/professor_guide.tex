%
% This file is part of PYCTS, the PY151 Credit Tracking System.
%
% PYCTS is free software: you can redistribute it and/or modify
% it under the terms of the GNU General Public License as published by
% the Free Software Foundation, either version 3 of the License, or
% (at your option) any later version.
%
% PYCTS is distributed in the hope that it will be useful,
% but WITHOUT ANY WARRANTY; without even the implied warranty of
% MERCHANTABILITY or FITNESS FOR A PARTICULAR PURPOSE.  See the
% GNU General Public License for more details.
%
% You should have received a copy of the GNU General Public License
% along with PYCTS.  If not, see <http://www.gnu.org/licenses/>.
%
% PYCTS and this file are Copyright 2011 by Mark Platek.
%

\documentclass[letterpaper]{article}

\usepackage[margin=2cm]{geometry}
\usepackage{graphicx}
\usepackage[small,compact]{titlesec}

% cause sections not to be numbered
\setcounter{secnumdepth}{-1} 

% set no paragraph indent and skip space between paragraphs
\setlength\parindent{0pt}
\setlength{\parskip}{8pt}

% no page numbers
\pagestyle{empty}

% PYCTS version
\newcommand{\softwareversion}{0.4.0.0}


\begin{document}

\begin{center}
{\bf {\huge PYCTS Primer for Professors } }

Software Version: \softwareversion

Prepared for: Department of Psychology, Clarkson University
\end{center}

\section{What is PYCTS?}
PYCTS is web software that allows students to quickly and easily check the number of research credits they've earned from participating in psychology research studies. You will be primarily reponsible for administrating the PYCTS software, including keeping the student roster up to date, adding new users, and making studies available to students.

\section{How do I add students to the roster?}
Before your research assistants can give out credits for students who have participated in research studies, you'll need to create a roster of all students enrolled in all sections of PY151. There are two ways to do this - you can add the students one at a time, or you can create a CSV file containing data for many students that can be uploaded to PYCTS. Either way, you'll need to go to the "Add Students" section. If you are going to use a CSV, please read Section 4.4 (Roster Management) of the User Manual carefully, as there are a few rules that you have to follow when creating the CSV.

\section{How do I add users?}
You can add users from the Admin Panel. Be careful when choosing the user level (RA or Professor), as only Professor-level users are capable of causing irreversible damage to the roster.

\section{What do I do with studies?}
An important feature of PYCTS involves tracking research studies. When students log in to check their credits, they can see a list of all the studies - this is much more convenient for students than traditional bulletin board flyers. Furthermore, by associating credits with studies, it's possible to collect statistics on how students are responding to the available studies.

To add a study, you'll need an IRB number and a flyer containing contact information (the flyer is required, and must be in PDF format). You can enter this information in the Studies Panel, using the title of the study as the description. You can delete any study as long as no credits have been given for it. If a study is closed to new applicants, you can change its visibility to make it so that students can't see the study or download its flyer.

\section{How do I give credits?}
Please review the Research Assistant Primer to learn how credits are added.

\section{How can I back up the data in PYCTS?}
It's possible to make a backup of the roster from the Backup Panel. This backup is simply a CSV file of all students in the roster and the number of credits they have, it doesn't preserve information about which students have participated in which studies. When you restore from a backup, PYCTS deletes all students and credits in the system, then uses the backup file to add new students. Each student gets a number of miscellaneous credits equal to the total number of credits they had when the backup was taken.

\section{How do I clean up at the end of the semester?}
Once a roster is no longer useful (e.g. after the semester has ended), you can wipe it from the Admin Panel. Doing so deletes all students and all credits that they have been given. This is also a good time to delete studies that will not be offered to students in the nest semester.






\end{document}











