%
% This file is part of PYCTS, the PY151 Credit Tracking System.
%
% PYCTS is free software: you can redistribute it and/or modify
% it under the terms of the GNU General Public License as published by
% the Free Software Foundation, either version 3 of the License, or
% (at your option) any later version.
%
% PYCTS is distributed in the hope that it will be useful,
% but WITHOUT ANY WARRANTY; without even the implied warranty of
% MERCHANTABILITY or FITNESS FOR A PARTICULAR PURPOSE.  See the
% GNU General Public License for more details.
%
% You should have received a copy of the GNU General Public License
% along with PYCTS.  If not, see <http://www.gnu.org/licenses/>.
%
% PYCTS and this file are Copyright 2011 by Mark Platek.
%

\documentclass[letterpaper]{article}

\usepackage[margin=2cm]{geometry}
\usepackage{graphicx}
\usepackage[small,compact]{titlesec}

% cause sections not to be numbered
\setcounter{secnumdepth}{-1} 

% set no paragraph indent and skip space between paragraphs
\setlength\parindent{0pt}
\setlength{\parskip}{8pt}

% no page numbers
\pagestyle{empty}

% PYCTS version
\newcommand{\softwareversion}{0.4.0.0}


\begin{document}

\begin{center}
{\bf {\huge PYCTS Primer for Research Assistants } }

Software Version: \softwareversion

Prepared for: Department of Psychology, Clarkson University
\end{center}

\section{What is PYCTS?}
PYCTS is web software that allows students to quickly and easily check the number of research credits they've earned from participating in psychology research studies. You will be in great part responsible for making sure that students' credit counts are kept up to date.

\section{Where is PYCTS?}
PYCTS can be found at the following URL:

{\tt http://clarkson.edu/projects/researchcredit}

You can log in using your Clarkson username and password. This is the same username and password you use to access other Clarkson web resources (e.g. Moodle and Peoplesoft).

\section{How do I add credits?}
There are two ways to add credits. The first and most convenient is through quick-add, the second it to do it directly.

To add credits through quick-add, make a selection of students in the roster table using the checkboxes at the far right. Then, use the dropdown menu at the top of the roster to select a study. Finally, click the "Add Credits to Selection" button to give the selected students credit for having performed the study. If you want, you can give a description of why the credits are being added - it's generally not necessary to do this, however.

To add credits directly, click on a student's last name in the roster to go to that student's page. From here, you can give study-associated credits in the same way that you gave credits through quick-add. If the situation calls for it, you can also give "miscellaneous" credits that are not associated with a study. This generally happens in the case of research papers or projects, but no matter the situation you must give a reason why the miscellaneous credits are being given.

\section{Whoops, I messed up. How do I delete credits?}
You can't {\it delete} credits per se; rather, you can {\it remove} them. Removed credits still show up in the student's credit listing (at the bottom of their page), but they don't affect the student's total number of credits, nor are they visible to students. To remove a credit, select it in the student's credit table, enter a reason for why the credit is being removed, and click the "Remove Credit" button.

\section{What are these statistics?}
As a Research Assistant, you are able to view aggregated statistics about the credits that have been given so far. Check it out if you're curious.

\end{document}













